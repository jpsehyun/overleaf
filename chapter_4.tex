\chapter{Methodology}
\label{sec:methodology}

This chapter enumerates and discusses the specific steps and activities that will be performed by the proponent to accomplish the objectives of the research. The discussion covers the activities from pre-proposal to Final Thesis Writing. 

Ethical issues surrounding the research and the steps to be taken by the research team to adhere to ethical guidelines and principles should be included when discussing each of the activities.

Research activities include inquiry, survey, research, brainstorming, canvassing, consultation, review, interview, observe, experiment, design, test, document, and other similar tasks. The following sections present an example set of activities and methods for conducting computing research. \textbf{\textcolor{red}{However, note that your group's actual research activities and methods will be different depending on your intended research contributions. You should consult closely with your thesis adviser for proper guidance. The following sections need not appear in your actual thesis proposal.}}

\section{Data Collection}

This section, also called "Data Gathering", focus on activities that will enable the researchers to understand their users and their context better, or to analyze the operating environment where the proposed software may be deployed. The types of activities would include interviews, observations, surveys, focus group discussions, and review of secondary data (reports, videos, manuals) to enable the researchers to identify user requirements; learn the existing processes, strategies, rules and guidelines; and/or build the datasets and knowledge resources needed in the study.

Ethical issues surrounding the collection of data from human participants, and from internet or 3rd party sources should be indicated. The steps to be taken by the proponents to ensure that ethical guidelines are followed should also be explained.

\section{Software Design and Implementation}

This section covers the design of data structures, knowledge bases, user interface, and algorithms. It includes implementation and related activities - tool selection, unit testing, integration testing, and function testing - that were covered in the course "Software Engineering".

Your Adviser may ask you to separate the Design and the Implementation sections as needed. 

You may also indicate a specific software development lifecycle model that you plan to adopt in developing your software, such as the Agile methodology.

For machine learning types of research, this section may be replaced with "Model Training".

\section{Validation}

This section contains the different types of validation activities that you will conduct to evaluate the performance of your model or algorithm or software. It may include user perception studies, end user acceptance testing, usability test, and model validation.

Ethical issues surrounding the collection of data from human participants should be indicated. The steps to be taken by the proponents to ensure that ethical guidelines are followed should also be explained.

This section should include a discussion of the following if working with human participants:
\begin{itemize}
    \item Participant Selection. Who are your target participants? How will they be selected? What data will be collected?
    \item Orientation. How will ethics be observed? What documents, e.g., validation protocol, will be shared with participants to orient them regarding the research and the validation procedure? Will there be an Informed Consent and Informed Assent? Are they found in the Appendix? Will there be any pre-tests or interviews to be administered prior to the actual experiment?
    \item Experiment Proper (aka Procedure). How long is the experiment or how long should the user use your software? Is it one-on-one, or one-to-many? What will you do while the experiment is ongoing - observe the participants using a checklist, guide the participants, record the interaction/usage?
    \item Debriefing. After the experiment, will you administer any post-tests or debriefing to solicit feedback from your participants? What are these?
\end{itemize}

\subsection{Calendar of Activities}

For the proposal, a Gantt chart showing the schedule of the activities should be added. You can create a table like below, or create a figure. For example:

Table \ref{tab:timetableactivities} shows a Gantt chart of the activities.  Each bullet represents approximately one week worth of activity.

\begin{table}[]
    \centering
    \begin{tabular}{l|c|c|c|c|c|c|c|c|c}
         \toprule
         & \multicolumn{6}{c|}{2021} & \multicolumn{3}{c}{2022} \\
         \textbf{Activity} & \textbf{Apr} & \textbf{May} & \textbf{Jun} & \textbf{Jul} & \textbf{Aug} & \textbf{Sep} & \textbf{Jan} & \textbf{Feb} & \textbf{Mar} \\
         \midrule
         Data Collection & **** & **** & & & & & & & \\
         \midrule
         Software Design & & **** & **** & **** & **** & **** & & & \\
         \midrule
         Validation & & & & & & **** & **** & **** & \\
         \bottomrule
    \end{tabular}
    \caption{Gantt Chart of Activities}
    \label{tab:timetableactivities}
\end{table}