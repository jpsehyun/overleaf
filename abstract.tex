%%%%%%%%%%%%%%%%%%%%%%%%%%%%%%%%%%%%%%%%%%%%%%%%%%%%%%%%%%%%%%%%%%%%%%%%%%%%%%%%%%%%%%%%%%%%%%%%%%%%%%
%
%   Filename    : abstract.tex 
%
%   Description : This file will contain your abstract.
%                 
%%%%%%%%%%%%%%%%%%%%%%%%%%%%%%%%%%%%%%%%%%%%%%%%%%%%%%%%%%%%%%%%%%%%%%%%%%%%%%%%%%%%%%%%%%%%%%%%%%%%%%

\begin{abstract}
History class has always been a trouble for students due to its nature in memorization for an exam-based classroom settings and requires an ability to visualize the events through word and pictures. Game-based learning such as simulations has risen to be effective in getting the attention of the students due to their inherent appeal and similarity to games; from which students and teens alike play worldwide. There has been numerous studies on game-based learning to overcome this problem but there are limited studies on simulations and integrating augmented reality technology into classroom settings, with few studies found within the Philippines'. This study aims to close the gap by developing mobile-based AR simulation application, and provide insights into its applicability and effectiveness within the Philippines’ educational system and additonally assess the effects of different AR components on student performance through an analysis of interactive elements, historical content representations, and immersive feature. 

\begin{flushleft}
\begin{tabular}{lp{4.25in}}
\hspace{-0.5em}\textbf{Keywords:}\hspace{0.25em} & Game-based learning, Augmented  reality, Education
\end{tabular}
\end{flushleft}
\end{abstract}
