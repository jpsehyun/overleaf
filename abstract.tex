%%%%%%%%%%%%%%%%%%%%%%%%%%%%%%%%%%%%%%%%%%%%%%%%%%%%%%%%%%%%%%%%%%%%%%%%%%%%%%%%%%%%%%%%%%%%%%%%%%%%%%
%
%   Filename    : abstract.tex 
%
%   Description : This file will contain your abstract.
%                 
%%%%%%%%%%%%%%%%%%%%%%%%%%%%%%%%%%%%%%%%%%%%%%%%%%%%%%%%%%%%%%%%%%%%%%%%%%%%%%%%%%%%%%%%%%%%%%%%%%%%%%

\begin{abstract}
History class has always been a trouble for students due to its nature in memorization for an exam-based classroom settings and requires an ability to visualize the events through words and pictures. There has been numerous studies on game-based learning to overcome this problem but there are limited studies on integrating augmented reality technology into game-based learning, and even less studies within the Philippines'.
\begin{itemize}
   \item \textbf{Motivation}. One to two sentence(s) describing the research domain or problem area.
   \item \textbf{Problem}. One to two sentence(s) describing the research challenge or opportunity to be addressed in your study
   \item \textbf{Contribution}. One to two sentence(s) describing your intended contribution or how you plan to address the research question 
   \item \textbf{Methodology}. One to two sentence(s) summarizing the approach or the manner in which you will carry out your study
   \item \textbf{Significant Results or Findings}. One to two sentence(s) describing your findings and result (This can be skipped during proposal stage) 
\end{itemize}

\begin{flushleft}
\begin{tabular}{lp{4.25in}}
\hspace{-0.5em}\textbf{Keywords:}\hspace{0.25em} & Keyword 1, keyword 2, keyword 3, keyword 4, etc. \textit{minimum of three relevant and descriptive keywords}\\
\end{tabular}
\end{flushleft}
\end{abstract}
