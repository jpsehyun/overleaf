%%%%%%%%%%%%%%%%%%%%%%%%%%%%%%%%%%%%%%%%%%%%%%%%%%%%%%%%%%%%%%%%%%%%%%%%%%%%%%%%%%%%%%%%%%%%%%%%%%%%%%
%
%   Filename    : chapter_1.tex 
%
%   Description : This file will contain your Research Description.
%                 
%%%%%%%%%%%%%%%%%%%%%%%%%%%%%%%%%%%%%%%%%%%%%%%%%%%%%%%%%%%%%%%%%%%%%%%%%%%%%%%%%%%%%%%%%%%%%%%%%%%%%%

\documentclass{article}
\usepackage{amsmath}
\usepackage{graphicx}
\usepackage{hyperref}

\title{Research Description}
\author{Your Name}
\date{\today}

\begin{document}

\maketitle

\chapter{Introduction}
\label{sec:intro}

\section{Background of the Study}
\label{sec:overview}

History is normally taught with traditional methods such as lectures and discussions. As it requires high levels of understanding, the youth often avoids history. There are other methods that lecturers utilize in teaching history, such as creative means like audio-visual presentations, software, and interactive means. Video games provide an excellent example of interactive learning. Not only is it popular among the new generation, but it grants freedom to choose how they may play the game.

\subsection{Students' Hardship in Classes}
There are multiple factors that make a student lose their interest and participation in typical classroom settings. Some factors include easy or difficult materials, lack of interest in a subject, and a lecture-based environment \cite{medium:mosley}. Out of all the problems causing students to disconnect from learning, test-driven classroom culture is one of the biggest factors impacting students' educational experience \cite{mora}. The students’ negative view of the classroom primarily focuses on multiple standardized tests they needed to prepare for and how classes were centered around these tests rather than students.

The same study by \cite{mora} showed how students were less bored and more engaged when classes were integrated with more interactive and hands-on activities such as poster-making and science experiments, rather than typical lecture-based classroom setups. This behavior of the students opens up the possibility of integrating game-based learning materials to enhance student involvement in the class and further improve their performance by creating a student-centered environment.

\subsection{Utilization of Augmented Reality with Affective Learning}
% TODO: Write this section

\section{Research Objectives}
% TODO: Write this section

\section{Scope and Limitations}
% TODO: Write this section

\section{Significance of the Study}
% TODO: Write this section

\end{document}
