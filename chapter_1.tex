%%%%%%%%%%%%%%%%%%%%%%%%%%%%%%%%%%%%%%%%%%%%%%%%%%%%%%%%%%%%%%%%%%%%%%%%%%%%%%%%%%%%%%%%%%%%%%%%%%%%%%
%
%   Filename    : chapter_1.tex 
%
%   Description : This file will contain your Research Description.
%                 
%%%%%%%%%%%%%%%%%%%%%%%%%%%%%%%%%%%%%%%%%%%%%%%%%%%%%%%%%%%%%%%%%%%%%%%%%%%%%%%%%%%%%%%%%%%%%%%%%%%%%%

\chapter{Introduction}
\label{sec:intro}    %--note: labels help you with hyperlink editing (using your IDE)
%%
%% --- 1.1 Background of the Study --- %%
%%

\section{Background of the Study}
\label{sec:overview}

%
%   NOTE: You have to delete/replace the unnecessary paragraphs with your own text.
%
History is normally taught with traditional methods such as lectures and discussions. As it requires high levels of understanding, the youth often avoids history. There are other methods that lecturers utilize in teaching history, such as creative means like audio-visual presentations, software, and interactive means. Video games provide an excellent example of interactive learning. Not only is it popular among the new generation, but it grants freedom to choose however they may play the game.

\subsection{Students' Hardship in Classes}
There are multiple factors which makes a student lose their interest and participation in a typical classroom settings. Some of the factors include easy or difficult materials, lack of interest in a subject, and a lecture-based environment \cite{medium:mosley}. Out of all the problems which causes the students to disconnect from the act of learning, test-driven classroom culture was one of the biggest factor which significantly impacts students' educational experience \cite{mora}. The students’ negative view of the classroom primarily focused on multiple standardized test they needed to prepare, and how classes were centered around these tests rather than students.

The same study by \cite{mora} showed how students were less bored and engaged when the classes were integrated with more interactive and hands-on activities such as poster-making and science experiments, rather than typical lecture based classroom set ups. This behavior of the students opens up to the possibility of integrating game-based learning material to enhance the involvement of the students to the class and further enhance their performances by creating a student-centered environment.


\subsection{Utilization of Augmented Reality with Affective Learning}
//TODO Write

%%
%% --- 1.3 Scope and Limitations --- %%
%%

\section{Scope and Limitations of the Research}
\label{sec:scopelimitations}

This section discusses the boundaries, with respect to the objectives, of the research and the constraints within which the research will be developed. Describe what is and is not included in the scope of your research, supported by your main research question and findings of previous studies. Do not use weak excuses such as the lack of time and/or knowledge to perform the research.

A good rule of thumb is to allocate one paragraph for each of your specific objectives that (1) contains a brief overview of the concept/theory and the purpose of doing the associated objective; and (2) includes a description of the scope/limitation of your study, and followed by brief purpose, rationale and/or justification for your decisions.

The following should also be indicated in your Scope and Limitations (in the appropriate paragraphs matching the objectives, or as a stand-alone paragraph):
\begin{itemize}
   \item The profile and demographics of your target participants
   \item Your data sources (i.e., new data, data from previous studies, data to be provided by some experts, data to be retrieved from social networks)
   \item The specific technology platform to be utilized
   \item The methods for collecting the data
   \item The coverage areas or locations
   \item The duration or time period (e.g., news articles for the year 2016-2017)
\end{itemize}


%%
%% --- 1.4 Significance of the Research --- %%
%%
\begin{comment}
\section{Significance of the Research}

This section explains why research must be done in this area.  It rationalizes the objective of the research with that of the stated problem. Avoid including sentences such as ``This research will be beneficial to the proponent/department/college'' as this is already an inherent requirement of all BS and MS thesis projects.  Focus on the research's contribution to the Computer Science field.

The following are guide questions that may help your formulate the significance of your research. 


%
% IPR acknowledgement: the following list of items are from Ethel Ong's slides on Significance of the Research
%
\begin{itemize}
\item  What is the relevance and contribution of your work to the computer science community? 

\begin{itemize} 
\item How does your technical contributions or empirical findings advance the field or grow our body of knowledge? 
\item If you built a prototype of an interaction technique, interface, library, tool, or system, what is value does it add compared to existing solutions? 
\end{itemize}

\item What will be your contributions to society in general? 
    \begin{itemize}
      \item How will your main stakeholders benefit from your technical contributions or empirical findings? 
      \item What are the positive social or economic impacts? 
   \end{itemize}
\end{itemize}
\end{comment}

\begin{comment}
If applicable, describe possible commercialization and/or innovation in your research.
\end{comment}

\section{Research Objectives}

\section{Scope and Limitations}

\section{Significance of the Study}