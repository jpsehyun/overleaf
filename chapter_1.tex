%%%%%%%%%%%%%%%%%%%%%%%%%%%%%%%%%%%%%%%%%%%%%%%%%%%%%%%%%%%%%%%%%%%%%%%%%%%%%%%%%%%%%%%%%%%%%%%%%%%%%%
%
%   Filename    : chapter_1.tex 
%
%   Description : This file will contain your Research Description.
%                 
%%%%%%%%%%%%%%%%%%%%%%%%%%%%%%%%%%%%%%%%%%%%%%%%%%%%%%%%%%%%%%%%%%%%%%%%%%%%%%%%%%%%%%%%%%%%%%%%%%%%%%

\chapter{Introduction}
\label{sec:intro}   
\section{Background of the Study}
\label{sec:overview}

History is normally taught with traditional methods such as lectures and discussions. As it requires high levels of understanding, the youth often avoids history. There are other methods that lecturers utilize in teaching history, such as creative means like audio-visual presentations, software, and interactive means. Video games provide an excellent example of interactive learning. Not only is it popular among the new generation, but it grants freedom to choose however they may play the game.

\subsection{Students' Hardship in Classes}
There are multiple factors which makes a student lose their interest and participation in a typical classroom settings. Some of the factors include easy or difficult materials, lack of interest in a subject, and a lecture-based environment \cite{medium:mosley}. Out of all the problems which causes the students to disconnect from the act of learning, test-driven classroom culture was one of the biggest factor which significantly impacts students' educational experience \cite{mora}. The students’ negative view of the classroom primarily focused on multiple standardized test they needed to prepare, and how classes were centered around these tests rather than students.

The same study by \cite{mora} showed how students were less bored and engaged when the classes were integrated with more interactive and hands-on activities such as poster-making and science experiments, rather than typical lecture based classroom set ups. This behavior of the students opens up to the possibility of integrating game-based learning material to enhance the involvement of the students to the class and further enhance their performances by creating a student-centered environment.


\subsection{Utilization of Augmented Reality with Affective Learning}
//TODO Write


\section{Research Objectives}
The project aims to develop an augmented reality simulation application that includes affective learning to enhance the user's understanding of historical events, specifically for this project - The Bataan Death March.


\section{Scope and Limitations}

\section{Significance of the Study}